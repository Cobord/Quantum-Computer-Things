\documentclass[11pt]{article}
\usepackage[margin=1in]{geometry} 
\geometry{letterpaper}   

\usepackage{amsmath}
\usepackage{amssymb,amsfonts,bbm,mathrsfs,stmaryrd}
\usepackage{url}

%%% Theorems and references %%%
\usepackage[amsmath,thmmarks]{ntheorem}
\usepackage{hyperref}
\usepackage{cleveref}

\theoremstyle{change}

\newtheorem{defn}[equation]{Definition}
\newtheorem{definition}[equation]{Definition}
\newtheorem{thm}[equation]{Theorem}
\newtheorem{theorem}[equation]{Theorem}
\newtheorem{prop}[equation]{Proposition}
\newtheorem{proposition}[equation]{Proposition}
\newtheorem{lemma}[equation]{Lemma}
\newtheorem{cor}[equation]{Corollary}
\newtheorem{conj}[equation]{Conjecture}
\newtheorem{conjecture}[equation]{Conjecture}
\newtheorem{exercise}[equation]{Exercise}
\newtheorem{example}[equation]{Example}

\theorembodyfont{\upshape}
\theoremsymbol{\ensuremath{\Diamond}}
\newtheorem{eg}[equation]{Example}
\newtheorem{remark}[equation]{Remark}

\theoremstyle{nonumberplain}

\theoremsymbol{\ensuremath{\Box}}
\newtheorem{proof}{Proof}

\qedsymbol{\ensuremath{\Box}}

\creflabelformat{equation}{#2(#1)#3} 

\crefname{equation}{equation}{equations}
\crefname{eg}{example}{examples}
\crefname{defn}{definition}{definitions}
\crefname{prop}{proposition}{propositions}
\crefname{thm}{Theorem}{Theorems}
\crefname{lemma}{lemma}{lemmas}
\crefname{cor}{corollary}{corollaries}
\crefname{remark}{remark}{remarks}
\crefname{section}{Section}{Sections}
\crefname{subsection}{Section}{Sections}

\crefformat{equation}{#2equation~(#1)#3} 
\crefformat{eg}{#2example~#1#3} 
\crefformat{defn}{#2definition~#1#3} 
\crefformat{prop}{#2proposition~#1#3} 
\crefformat{thm}{#2Theorem~#1#3} 
\crefformat{lemma}{#2lemma~#1#3} 
\crefformat{cor}{#2corollary~#1#3} 
\crefformat{remark}{#2remark~#1#3} 
\crefformat{section}{#2Section~#1#3} 
\crefformat{subsection}{#2Section~#1#3} 

\Crefformat{equation}{#2Equation~(#1)#3} 
\Crefformat{eg}{#2Example~#1#3} 
\Crefformat{defn}{#2Definition~#1#3} 
\Crefformat{prop}{#2Proposition~#1#3} 
\Crefformat{thm}{#2Theorem~#1#3} 
\Crefformat{lemma}{#2Lemma~#1#3} 
\Crefformat{cor}{#2Corollary~#1#3} 
\Crefformat{remark}{#2Remark~#1#3} 
\Crefformat{section}{#2Section~#1#3} 
\Crefformat{subsection}{#2Section~#1#3} 


\numberwithin{equation}{section}


%%% Tikz stuff %%%

\usepackage{tikz}
\tikzset{dot/.style={circle,draw,fill,inner sep=1pt}}
\usepackage{braids}
\usepackage{tqft}
\usetikzlibrary{tqft}
\usetikzlibrary{cd}
\usetikzlibrary{arrows}
%%% Letters, Symbols, Words %%%

\newcommand\Aa{{\cal A}}
\newcommand\Oo{{\cal O}}
\newcommand\Uu{{\cal U}}
\newcommand\NN{{\mathbb N}}
\newcommand\RR{{\mathbb R}}
\newcommand\Ddd{\mathscr{D}}
\renewcommand{\d}{{\,\rm d}}
\newcommand\T{{\rm T}}

\newcommand\mono{\hookrightarrow}
\newcommand\sminus{\smallsetminus}
\newcommand\st{{\textrm{ s.t.\ }}}
\newcommand\ket[1]{\mid #1 \rangle}
\newcommand\bra[1]{\langle #1 \mid}
\newcommand\setof[1]{\{ #1 \}}
\newcommand\lt{<}
\newcommand\abs[1]{ \mid #1 \mid }
\newcommand\conjbar[1]{\overline{#1}}

\DeclareMathOperator{\Aut}{Aut}
\DeclareMathOperator{\dVol}{dVol}
\DeclareMathOperator{\ev}{ev}
\DeclareMathOperator{\fiber}{fiber}
\DeclareMathOperator{\GL}{GL}
\DeclareMathOperator{\id}{id}
\DeclareMathOperator{\sign}{sign}
\DeclareMathOperator{\tr}{tr}
\setlength{\parskip}{1em}

\title{Group Rewriting}
\author{Ammar Husain}

\begin{document}
\maketitle

\section{Clifford Group}

\subsection{Pauli Group}

\begin{definition}[$P_n$]
The Pauli Group on $n$ qubits is the group generated by $\sigma_x$ $\sigma_y$ $\sigma_z$ and $i$ operating on each of $n$ qubits. As a group it is isomorphic to $(D_4 \rtimes C_2)^n$
\end{definition}

\subsection{Clifford Group}

\begin{definition}[$C_n$]
The normalizer of $P_n$ in $U(2^n)$.
\end{definition}

\begin{lemma}
It is generated by the generators of $P_n$ along with the following $?$.
\end{lemma}

\subsection{Cliff+T}

\begin{theorem}[Universality]
\end{theorem}

\begin{theorem}[Solovay-Kitaev]
\end{theorem}

As a corolary of universality we know that we can take an arbitrary rotation of one qubit $SU(2)$ to an approximation using only Cliff+T gates. A desired error bound must be given. An algorithm for this is given by\\
\url{https://arxiv.org/pdf/1403.2975.pdf}\\
but it is faster than Solovay-Kitaev which is more general.

\section{Coxeter Groups}

2,3 qubit relations with weyl exceptional other fun groups\\
\url{https://hal.inria.fr/file/index/docid/420456/filename/E8Weyl.pdf}\\
\url{https://arxiv.org/pdf/0807.3650.pdf}\\

Rewriting system for coxeter group\\
\url{https://ac.els-cdn.com/0022404994900191/1-s2.0-0022404994900191-main.pdf?_tid=1f07609c-04b2-11e8-8ed1-00000aab0f6c&acdnat=1517202550_f41a15336d6c1eb1d880008e0214ec41}

Words with intervening property for a quick check of reducedness. Outputs a witness of nonreducedness. But no witness found does not imply reduced.\\
Given an expression and a coxeter graph, algorithm to find a segment of the word to show it is not reduced/null if this is inconclusive\\
Then feed that into the reducing algorithm. Repeat until intervening property says the expression might be reduced.\\
\url{http://emis.ams.org/journals/EJC/Volume_17/PDF/v17i1n9.pdf}

Algorithm for reducing words in Coxeter group\\
\url{https://mathoverflow.net/questions/109071/algorithm-for-reducing-words-in-a-coxeter-group}

\begin{definition}[Automatic Group]
A group $G$ with generators $A$ is a set of finite state automata. There is a word acceptor which inputs a group element and a word and outputs accept or not. It accepts at least one word for every $g$. There are also $\abs{A}+1$ multipliers which input a pair $w_1,w_2$ of words that are each accepted and outputs accept or not based on whether $w_1 a = w_2$ where $a \in A \bigcup 1$ is which multiplier we are using.\\
The word acceptor tells you whether the word is reduced and the multiplier can now check equality on these reduced form with less difficulty.
\end{definition}

\begin{example}
Finite groups, Euclidean groups, finitely generated Coxeter groups are all automatic.
\end{example}

\begin{theorem}
Automatic groups have the property that any given word can be put into canonical form in quadratic time. Putting any two words into canonical form and then checking equality solves the word problem in quadratic time.
\end{theorem}

\subsection{X,Y,Z,H and Swaps}

\begin{lemma}
The group generated by these gates has the following relations.

\begin{eqnarray*}
( X_i Y_i )^4 &=& 1\\
( X_i Z_i )^4 &=& 1\\
( Y_i Z_i )^4 &=& 1\\
( H_i X_i )^8 &=& 1\\
( H_i Y_i )^4 &=& 1\\
( H_i Z_i )^8 &=& 1\\
(H_i S_{i,i+1})^4 &=& 1\\
(X_i S_{i,i+1})^4 &=& 1\\
(Y_i S_{i,i+1})^4 &=& 1\\
(Z_i S_{i,i+1})^4 &=& 1\\
(H_{i+1} S_{i,i+1})^4 &=& 1\\
(X_{i+1} S_{i,i+1})^4 &=& 1\\
(Y_{i+1} S_{i,i+1})^4 &=& 1\\
(Z_{i+1} S_{i,i+1})^4 &=& 1\\
(S_{i,i+1} S_{i+1,i+2})^3 &=& 1\\
\end{eqnarray*}

There may be additional relations, but that would just be more quotienting, so reducing a word using only these relations would still be valid.
\end{lemma}

\begin{proof}
See the Mathematica notebook
\end{proof}

Make a picture of the Coxeter graph with $i=1,2,3$ for 3 qubits. 3 colors of edges for order $3$, $4$ and $8$. As usual no edge for order $2$. For more qubits just extend this graph by adding more basic units of the same structure.

The picture shows the big Coxeter group that is very redundant and has many more additional relations. One way to cut this down is to make all the 1-qubit gates act on qubit 1 and all 2 qubit gates on 1 and 2. This can be done by using nonquadratic relations with the symmetric group to put everything into position. This gives fewer generators and less higher order relations, but will make the words solved for under group rewriting have a restricted vocabulary. So this is a choice to trade off.

\section{Finite Complete Rewriting Systems}

Surface groups but more importantly for here theorem about short exact sequences\\
\url{https://arxiv.org/pdf/math/9611205.pdf}

Rewriting system for coxeter group\\
\url{https://ac.els-cdn.com/0022404994900191/1-s2.0-0022404994900191-main.pdf?_tid=1f07609c-04b2-11e8-8ed1-00000aab0f6c&acdnat=1517202550_f41a15336d6c1eb1d880008e0214ec41}

Gap for all finitely presented groups\\
\url{http://doc.sagemath.org/html/en/reference/groups/sage/groups/finitely_presented.html}

\section{Unsorted}

\url{https://arxiv.org/pdf/1509.02004.pdf}\\
\url{https://arxiv.org/pdf/1701.05200.pdf}

\end{document}