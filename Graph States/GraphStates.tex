\documentclass[11pt]{article}
\usepackage[margin=1in]{geometry} 
\geometry{letterpaper}   

\usepackage{amsmath}
\usepackage{amssymb,amsfonts,bbm,mathrsfs,stmaryrd}
\usepackage{url}

%%% Theorems and references %%%
\usepackage[amsmath,thmmarks]{ntheorem}
\usepackage{hyperref}
\usepackage{cleveref}

\theoremstyle{change}

\newtheorem{defn}[equation]{Definition}
\newtheorem{definition}[equation]{Definition}
\newtheorem{thm}[equation]{Theorem}
\newtheorem{theorem}[equation]{Theorem}
\newtheorem{prop}[equation]{Proposition}
\newtheorem{proposition}[equation]{Proposition}
\newtheorem{lemma}[equation]{Lemma}
\newtheorem{cor}[equation]{Corollary}
\newtheorem{conj}[equation]{Conjecture}
\newtheorem{conjecture}[equation]{Conjecture}
\newtheorem{exercise}[equation]{Exercise}
\newtheorem{example}[equation]{Example}

\theorembodyfont{\upshape}
\theoremsymbol{\ensuremath{\Diamond}}
\newtheorem{eg}[equation]{Example}
\newtheorem{remark}[equation]{Remark}

\theoremstyle{nonumberplain}

\theoremsymbol{\ensuremath{\Box}}
\newtheorem{proof}{Proof}

\qedsymbol{\ensuremath{\Box}}

\creflabelformat{equation}{#2(#1)#3} 

\crefname{equation}{equation}{equations}
\crefname{eg}{example}{examples}
\crefname{defn}{definition}{definitions}
\crefname{prop}{proposition}{propositions}
\crefname{thm}{Theorem}{Theorems}
\crefname{lemma}{lemma}{lemmas}
\crefname{cor}{corollary}{corollaries}
\crefname{remark}{remark}{remarks}
\crefname{section}{Section}{Sections}
\crefname{subsection}{Section}{Sections}

\crefformat{equation}{#2equation~(#1)#3} 
\crefformat{eg}{#2example~#1#3} 
\crefformat{defn}{#2definition~#1#3} 
\crefformat{prop}{#2proposition~#1#3} 
\crefformat{thm}{#2Theorem~#1#3} 
\crefformat{lemma}{#2lemma~#1#3} 
\crefformat{cor}{#2corollary~#1#3} 
\crefformat{remark}{#2remark~#1#3} 
\crefformat{section}{#2Section~#1#3} 
\crefformat{subsection}{#2Section~#1#3} 

\Crefformat{equation}{#2Equation~(#1)#3} 
\Crefformat{eg}{#2Example~#1#3} 
\Crefformat{defn}{#2Definition~#1#3} 
\Crefformat{prop}{#2Proposition~#1#3} 
\Crefformat{thm}{#2Theorem~#1#3} 
\Crefformat{lemma}{#2Lemma~#1#3} 
\Crefformat{cor}{#2Corollary~#1#3} 
\Crefformat{remark}{#2Remark~#1#3} 
\Crefformat{section}{#2Section~#1#3} 
\Crefformat{subsection}{#2Section~#1#3} 


\numberwithin{equation}{section}


%%% Tikz stuff %%%

\usepackage{tikz}
\tikzset{dot/.style={circle,draw,fill,inner sep=1pt}}
\usepackage{braids}
\usepackage{tqft}
\usetikzlibrary{tqft}
\usetikzlibrary{cd}
\usetikzlibrary{arrows}
%%% Letters, Symbols, Words %%%

\newcommand\Aa{{\cal A}}
\newcommand\Oo{{\cal O}}
\newcommand\Uu{{\cal U}}
\newcommand\NN{{\mathbb N}}
\newcommand\RR{{\mathbb R}}
\newcommand\Ddd{\mathscr{D}}
\renewcommand{\d}{{\,\rm d}}
\newcommand\T{{\rm T}}

\newcommand\mono{\hookrightarrow}
\newcommand\sminus{\smallsetminus}
\newcommand\st{{\textrm{ s.t.\ }}}
\newcommand\ket[1]{\mid #1 \rangle}
\newcommand\bra[1]{\langle #1 \mid}
\newcommand\setof[1]{\{ #1 \}}
\newcommand\lt{<}
\newcommand\abs[1]{ \mid #1 \mid }
\newcommand\conjbar[1]{\overline{#1}}

\DeclareMathOperator{\Aut}{Aut}
\DeclareMathOperator{\dVol}{dVol}
\DeclareMathOperator{\ev}{ev}
\DeclareMathOperator{\fiber}{fiber}
\DeclareMathOperator{\GL}{GL}
\DeclareMathOperator{\id}{id}
\DeclareMathOperator{\sign}{sign}
\DeclareMathOperator{\tr}{tr}


\title{Graph States}
\author{Ammar Husain}

\begin{document}
\maketitle

\section{Graph States}

\begin{definition}[Graph State]
Let $G$ be an undirected graph with $V$ vertices.

\begin{eqnarray*}
\ket{\psi_G} &\equiv& \prod_{(a,b) \in E} CZ_{a,b} \ket{+}^{\otimes V}
\end{eqnarray*}

This is well founded because $CZ_{a,b} = CZ_{b,a}$ so the choice of orientation for $(a,b) \in E$ does not matter. Also they commute so the order of the product does not have to be specified.
\end{definition}

\begin{lemma}
For $K_i = X_i \otimes_{j \in N(i)} Z_j$, $K_i \ket{\psi_G} = (+1)\ket{\psi_G}$. $K_i^2 = Id$, and they commute. So they form an independent set in the sense of the Coxeter graph as the general program described in the Coxeter compiler paper.
\end{lemma}

\section{Hypergraph States}

\begin{definition}[$k$-Uniform Hypergraph]
A hypergraph such that each hyperedge contains exactly $k$ vertices. If $k=2$, that is a usual graph.
\end{definition}

\begin{definition}[Uniform Hypergraph State]
Let $G$ be a $k$-uniform hypergraph on $V$ vertices.

\begin{eqnarray*}
\ket{\psi_G} &\equiv& \prod_{(a_1,a_2 \cdots a_k) \in E} CZ_{a_1,a_2 \cdots a_k} \ket{+}^{\otimes V}
\end{eqnarray*}

where $CZ_{a_1,a_2 \cdots a_k}$ acts as $-1$ only on $\ket{1}^{\otimes k}$ and $1$ otherwise. Again the symmetry means there is no impact of the choice of ordering $a_1 \cdots a_k$ necessary in the hyperedge.

\end{definition}

\begin{lemma}
For $K_i = X_i \otimes \prod_{(i,i_2 \cdots i_k) \in E} CZ_{i_2 \cdots i_k}$, $K_i \ket{\psi_G} = (+1)\ket{\psi_G}$. Again $K_i^2 = Id$, and they commute. Again they form an independent set in the sense of the Coxeter graph as the general program described in the Coxeter compiler paper.
\end{lemma}

\begin{definition}[Hypergraph state]
\end{definition}

\begin{proof}
\url{https://arxiv.org/pdf/1211.5554.pdf}
\url{https://arxiv.org/pdf/1612.06418.pdf}
\end{proof}

\section{Bayesian DAG States}

\begin{definition}
Let $G$ be a DAG where the vertices $v_i$ represent random variables on $d_i$ possibilities. So the entire Hilbert space is $\bigotimes_{i=0}^{V-1} \mathbb{C}^{d_i}$ The edges indicate causation.

Let $b$ be vertex and $\setof{a_1 \cdots a_n}$ be it's possibly empty set of immediate predecessors $U_{\setof{a},b}$ is then chosen to be a unitary satisfying

\begin{eqnarray*}
U_{\setof{a},b} \ket{e_{i_1}} \otimes \cdots \ket{e_{i_n}} \otimes \ket{0} &=& \sqrt{P_j} \ket{e_{i_1}} \otimes \cdots \ket{e_{i_n}} \otimes \ket{j}\\
P_j &=& P( b = j \mid a_1 = i_1 \cdots a_n = i_n )\\
\end{eqnarray*}

Initialize in the state $\ket{0}^{\otimes V}$. Do a topological sort on $G$. Then apply the operators $U_{\varnothing, b}$ for the bottom of the poset. Then build up from there with $U_{\setof{a},b}$

\end{definition}

If you wish to ask a question conditioned on some variables like the random variable for vertex $v_j$ is in state $i_j$, perform amplitude amplification first. The ratios between the terms with $v_j = i_j$ remain the same amongst each other. That is what one is probing anyways. Relative likelihoods.

\end{document}